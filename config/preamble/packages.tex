
\usepackage{import}          % Establish input relative to a directory
\usepackage{verbatim}        % Reimplementation of and extensions to LATEX verbatim. Useful for text-blocks and code-blocks

\usepackage[en-GB]{datetime2}

% Structure & Layout
\usepackage{calc,soul,fourier} % Arithmetic in LATEX commands. Required for Chapters
\usepackage{rotfloat} % Rotate floats. E.g. tables, figures. Used for Chapter number box
\usepackage{parskip} % Layout with zero \parindent, non-zero \parskip - Anthony
%\usepackage{balance} % Balanced two-column mode
%\usepackage{float}   % Improved interface for floating objects
\usepackage{mdframed} % Frames and boxes
\usepackage{fancybox}
\usepackage{pdflscape}
\usepackage{wrapfig}
\usepackage{lscape}
\usepackage{rotating}

% Fonts & Symbols
\usepackage{newlfont}     % Helpful package for fonts and symbols. Required by memoir.
\usepackage{calligra}     % Handwriting style. Used for dedication


% Colour, Images, & Graphics
\usepackage[table]{xcolor} % Driver-independent color extensions for LATEX and pdfLATEX
\usepackage{tikz}          % For absolute positioning of images. Used on declaration page

\usetikzlibrary{shapes.geometric, arrows.meta, positioning, calc, fit}
\usepackage{pgfplots}
% Citation, referencing
\usepackage[colorlinks, linkcolor = black, citecolor = black, filecolor = black,
urlcolor = black]{hyperref} % Hypertext. E.g. Creates hyperlinks in cross references

\usepackage[acronyms,toc,nonumberlist]{glossaries-extra}

% Tables

\usepackage{multirow}                  % Create tabular cells spanning multiple rows

\usepackage{tablefootnote}              % Permit footnotes in tables


%algorithms
\usepackage{algorithm}
\usepackage{algorithmic}

% Lists
\usepackage[shortlabels]{enumitem}
\usepackage{enumerate}


\DeclareOldFontCommand{\rm}{\normalfont\rmfamily}{\mathrm}

%subfigures
\usepackage{subcaption}

%code listings
\usepackage{listings}
\definecolor{codegreen}{rgb}{0,0.6,0}
\definecolor{codegray}{rgb}{0.5,0.5,0.5}
\definecolor{codepurple}{rgb}{0.58,0,0.82}
\definecolor{backcolour}{rgb}{0.95,0.95,0.92}

\lstdefinestyle{mystyle}{
    backgroundcolor=\color{backcolour},   
    commentstyle=\color{codegreen},
    keywordstyle=\color{magenta},
    numberstyle=\tiny\color{codegray},
    stringstyle=\color{codepurple},
    basicstyle=\ttfamily\footnotesize,
    breakatwhitespace=false,         
    breaklines=true,                 
    captionpos=b,                    
    keepspaces=true,                 
    numbers=left,                    
    numbersep=5pt,                  
    showspaces=false,                
    showstringspaces=false,
    showtabs=false,                  
    tabsize=2
}
\lstset{style=mystyle}


%if using hindi font such as for quotes
%\usepackage{fontspec}
%\newfontfamily\hindifont{Noto Sans Devanagari}[Script=Devanagari] % Use any Devanagari font on your system

%braket notation in QM
\usepackage{braket}




